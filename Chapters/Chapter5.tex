% Chapter Template

\chapter{Conclusiones} % Main chapter title

\label{Chapter5} % Change X to a consecutive number; for referencing this chapter elsewhere, use \ref{ChapterX}

En este capítulo se destacan los principales aportes del trabajo realizado y se listan los logros obtenidos.  Asimismo, se documentan las técnicas que resultaron  útiles para la ejecución del proyecto.  Por otra parte, se deja constancia de las metas que no pudieron ser alcanzadas junto con las respectivas causas y se identifican las líneas de acción a futuro.
%----------------------------------------------------------------------------------------

%----------------------------------------------------------------------------------------
%	SECTION 1
%----------------------------------------------------------------------------------------

\section{Conclusiones generales }

%La idea de esta sección es resaltar cuáles son los principales aportes del trabajo realizado y cómo se podría continuar. Debe ser especialmente breve y concisa. Es buena idea usar un listado para enumerar los logros obtenidos.

En este trabajo se completó una primera iteración en el ciclo de diseño e implementación de un sistema embebido para el control de una estación de monitoreo de ruido ambiente submarino.  

Se pudo desarrollar un \textit{firmware} multicore modular que sienta las bases para una segunda iteración donde se diseñen e implementen tanto los componentes que quedaron fuera del alcance de esta etapa, así como también los que estaban contemplados pero no pudieron implementarse.

Los principales aportes de este trabajos son:

\begin{itemize}
	\item Utilización de una metodología de desarrollo basada en control de versiones sistemática y robusta.
	\item Desarrollo de una arquitectura modular multicore que permite el intercambio de funcionalidades entre procesadores de forma flexible.
	\item Implementación de mecanismos de comunicación y sincronización entre procesadores, basados en interrupciones cruzadas y colas de mensajes.
	\item Implementación de mecanismos de control y despacho de tareas y eventos.
	\item Desarrollo de cuatro módulos con funcionalidad de adquisición de datos, almacenamientos de datos, HMI y control del sistema, respectivamente. 
%	\item Implementación de mecanismos de control y autochequeo para aumentar la seguridad de operación del sistema.
	\item Documentación completa de ingeniería de detalle de los módulos desarrollados.
	\item Documentación completa de testing.
	\item Documentación completa de trazabilidad entre requerimientos, ingeniería de detalle y testing.
\end{itemize} 

\subsection{Técnicas útiles}
\label{subsec:tecnicas_utiles}

Resultaron particularmente útiles las técnicas de gestión de proyecto empleadas en la planificación de este trabajo.  Mediante el desglose del proyecto en tareas y la articulación de éstas en diagramas de \textit{Activity on Node} y \textit{Gantt}, se puedo definir una metodología de seguimiento y control que permitió detectar retrasos en el avance del proyecto. Se aplicaron medidas de mitigación \textit{ad-hoc} que consistieron en:

\begin{itemize}
	\item el incremento de la dedicación horaria destinada al proyecto,
	\item el empleo de funciones \textit{mock} o maquetadas para algunas características planificadas que no llegaron a implementarse.
	\item la priorización del requerimiento implícito de fecha de finalización por sobre la cantidad de sensores operativos, en última instancia.
\end{itemize}  

%Este plan de acción neutralizó parcialmente los efectos negativos sobre el resultado del proyecto al costo de no cumplir .

Dentro de la gestión de riesgos no se contempló explícitamente la posibilidad de imprevistos como los que efectivamente se manifestaron.  Una análisis posterior de los hechos muestra que los retrasos responden principalmente a dos fuentes:

\begin{enumerate}
	\item un evento fortuito, único e irrepetible, en el ámbito laboral, que sobrecargó la dedicación horaria a tareas ajenas al proyecto durante gran parte de tiempo planificado para su realización e incluyó día de embarque.
	\item la incorporación de nuevas responsabilidades en el campo de la docencia.
\end{enumerate}

La primera fuente constituye un hecho aislado y, por su carácter de imprevisible, resulta razonable que haya quedado fuera del alcance de la gestión de riesgos.  

La segunda fuente implica un error de cálculo en la relación entre horas disponibles y la carga horaria de las tareas de docencia asumidas.  Esto último será tenido en cuenta como aprendizaje para futuros proyecto.

En términos generales, fue correcta la valoración de severidad y tasa de ocurrencia para todos los riesgos identificados; de los riesgos originalmente previstos:

\begin{enumerate}
	\item No contar a tiempo con el anemómetro.
	\item No contar con drivers para controlar el termómetro digital.
	\item No contar con el becario PIDDEF a tiempo para que pueda participar en el proyecto.
	\item Pérdida, robo o destrucción total o parcial de la placa CIAA-NXP y/o el anemómetro
	\item Cancelación del proyecto PIDDEF del Ministerio de Defensa asociado.
\end{enumerate}

\noindent sólo se manifestó íntegramente el riesgo número 3, que acertadamente fue valorado con una probabilidad de la ocurrencia alta. En este sentido, resultó adecuada la estrategia de planificar las tareas para ser llevadas a cabo sin la colaboración de un becario, por lo cual no se produjo ningún impacto negativo en el proyecto. 

Respecto al riesgo número 5, si bien el proyecto no fue cancelado, el hecho de contar con un subsidio trianual como el otorgado por el programa PIDDEF \citep{PIDDEF}, que haya coincidido con el cambio de gestión en la administración pública ocasionó retrasos y sobrecostos significativos que afortunadamente pudieron ser sobrellevados.

%En términos generales, fue correcta la valoración de severidad y tasa de ocurrencia para todos los riesgos identificados.

\subsection{Metas alcanzadas}
\label{subsec:metasalcanzadas}

Se recopila en la tabla \ref{tab:reqs_alcanzados} el estado de los requerimientos, indicando mediante un tilde y color verde los que fueron alcanzados satisfactoriamente.  Asimismo, se indica con una X y color rojo los requerimientos que no pudieron lograrse.

% Please add the following required packages to your document preamble:
% \usepackage{booktabs}
% \usepackage[table,xcdraw]{xcolor}
% If you use beamer only pass "xcolor=table" option, i.e. \documentclass[xcolor=table]{beamer}
\begin{table}[!htpb]
\centering
\caption{Requerimientos alcanzados.}
\label{tab:reqs_alcanzados}
\begin{tabular}{@{}lc@{}}
\toprule
%\rowcolor[HTML]{9B9B9B} 
\textbf{Requerimiento} & \textbf{Estado} \\ \midrule
\rowcolor[HTML]{EFEFEF} 
1. Requerimientos de documentación &  \\
 &  \\
 \rowcolor[HTML]{9AFF99} 
\begin{tabular}[c]{@{}l@{}}1.1 Se debe generar un Memoria Técnica con la documentación \\ de ingeniería detallada.\end{tabular} &  \checkmark \\
 &  \\
\rowcolor[HTML]{9AFF99} 
1.2 Se debe generar un documento de casos de prueba. &  \checkmark \\
 &  \\
\rowcolor[HTML]{EFEFEF} 
2. Requerimientos funcionales del sistema &  \\
 &  \\
\rowcolor[HTML]{9AFF99} 
\begin{tabular}[c]{@{}l@{}}2.1 El sistema debe adquirir datos de un array de sensores de \\ temperatura a intervalos regulares con un período de adquisición \\ seleccionable.\end{tabular} &  \checkmark \\
 &  \\
 \rowcolor[HTML]{FFCCC9} 
\begin{tabular}[c]{@{}l@{}}2.2 El sistema debe adquirir datos de un anemómetro a intervalos \\ regulares con un período de adquisición seleccionable.\end{tabular} &  X \\
 &  \\
\rowcolor[HTML]{9AFF99} 
\begin{tabular}[c]{@{}l@{}}2.3 El sistema debe almacenar los datos de temperatura y velocidad\\ de viento adquiridas junto con una marca de tiempo identificatoria \\ en un medio físico no volátil.\end{tabular} & \checkmark \\
 &  \\
\rowcolor[HTML]{9AFF99} 
\begin{tabular}[c]{@{}l@{}}2.4 El sistema debe poder operar con dos perfiles de consumo de \\ energía máximo desempeño y mínimo consumo de energía.\end{tabular} & \checkmark \\
 &  \\
 \rowcolor[HTML]{9AFF99} 
\begin{tabular}[c]{@{}l@{}}2.5 El sistema debe contar con una interfaz serie cableada que \\ permita realizar operaciones de configuración y mantenimiento.\end{tabular} & \checkmark \\
 &  \\
\rowcolor[HTML]{EFEFEF} 
3. Requerimientos de verificación &  \\
 &  \\
 \rowcolor[HTML]{9AFF99} 
\begin{tabular}[c]{@{}l@{}}3.1 Se debe generar una matriz de trazabilidad entre la Memoria\\ Técnica y los requerimientos.\end{tabular} & \checkmark \\
 &  \\
 \rowcolor[HTML]{9AFF99} 
\begin{tabular}[c]{@{}l@{}}3.2 Se debe generar una matriz de trazabilidad entre las pruebas \\ de integración y los requerimientos.\end{tabular} & \checkmark \\
 &  \\
\rowcolor[HTML]{EFEFEF} 
4. Requerimientos de validación &  \\
 &  \\
 \rowcolor[HTML]{9AFF99} 
\begin{tabular}[c]{@{}l@{}}4.1 Se debe generar una matriz de trazabilidad entre el documento\\ de casos de prueba y los requerimientos.\end{tabular} & \checkmark \\ \bottomrule
\end{tabular}
\end{table}

A los fines prácticos de terminar el trabajo en la fecha pactada, no fue posible implementar el control del anemómetro.  Esto no responde a una dificultad técnica sino a hechos imprevistos en la planificación que sobrecargaron la dedicación horaria a actividades ajenas al proyecto e imposibilitaron el cumplimiento del requisito asociado a este sensor como fuera mencionado en la subsección \ref{subsec:tecnicas_utiles}.

%----------------------------------------------------------------------------------------
%	SECTION 2
%----------------------------------------------------------------------------------------
\section{Próximos pasos}

Se indican en esta sección las líneas de acción más inmediatas para continuar la implementación del sistema e incorporar nuevas características a la estación de medición en desarrollo.  Asimismo se establece un orden de prioridad para las acciones identificadas, a saber:

\begin{enumerate}
  \item Implementar el control del anemómetro y completar los requerimientos previstos para esta etapa del proyecto.
  \item Reemplazar las funciones \textit{mock} que interactuan con la interfaz del sistema y así completar la implementación de todos los estados de las máquinas de estados finitos principales de cada módulo.
  \item Incorporar más opciones de configuración a cada módulo.  Para este fin, se contempla el pasaje por referencia de una estructura de configuración diferenciada para cada módulo que contenga los distintos parámetros configurables.  Asimismo, se debe reformular el diseño de la interfaz máquina-hombre para permitir la toma de dichos parámetros.
  \item Analizar la mejor forma de aprovechar las características asimétricas de los dos procesadores que posee la plataforma CIAA-NXP en cuanto a distribución de los módulos entre los procesadores.  La arquitectura desarrollada permite que el mismo código sea descargado a ambos procesadores, debiendo elegirse en un archivo de configuración qué módulo se ejecuta en qué procesador para que no ocurran colisiones a la hora de acceder a los recursos.  Esto permite explorar alternativas que aumenten la seguridad del sistema ante eventuales fallas de un procesador, existiendo redundancia de otro procesador que pueda retomar la ejecución de las tareas caídas.  Asimismo, resulta de interés evaluar el desempeño del sistema en cuanto a consumo eléctrico, disipación térmica y tiempo de cómputo, entre otras características, en diferentes configuraciones de reparto de módulos entre los procesadores.
  \item Incorporar al modelo de desarrollo un servidor de integración continua \citep{CI} para automatizar pruebas estáticas y dinámicas sobre el código.  
  \item Incorporar los subsistemas contemplados para futuras etapas en la figura \ref{fig:diagramaBloques} y que fueron considerados fuera del alcance en este proyecto, en forma de nuevos módulos de la estación, siguiendo el mismo patrón de diseño presentado en este trabajo.
  \item Diseñar y fabricar un poncho para la CIAA-NXP que permita conectar y desconectar en forma robusta las diferentes entradas y salidas del sistema de forma de poder operar la estación en condiciones de campo.
  \item Realizar pruebas de campo. 
\end{enumerate}


