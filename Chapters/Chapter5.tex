% Chapter Template

\chapter{Conclusiones} % Main chapter title

\label{Chapter5} % Change X to a consecutive number; for referencing this chapter elsewhere, use \ref{ChapterX}

En este capítulo se destacan los principales aportes del trabajo realizado y se listan los logros obtenidos.  Asimismo, se documentan las técnicas que resultaron  útiles para la ejecución del proyecto.  Por otra parte, se deja constancia de las metas que no pudieron ser alcanzadas junto con las respectivas causas, de modo que sirva como experiencia para futuros proyectos.  Finalmente, se identifican las líneas de acción a futuro y posibles mejoras para continuar el trabajo más adelante.
%----------------------------------------------------------------------------------------

%----------------------------------------------------------------------------------------
%	SECTION 1
%----------------------------------------------------------------------------------------

\section{Conclusiones generales }

%La idea de esta sección es resaltar cuáles son los principales aportes del trabajo realizado y cómo se podría continuar. Debe ser especialmente breve y concisa. Es buena idea usar un listado para enumerar los logros obtenidos.

En este trabajo se logró completar una primera iteración en el ciclo de diseño e implementación de un sistema embebido para el control de una estación de monitorio de ruido ambiente submarino.  

Se pudo desarrollar un \textit{firmware} multicore modular que sienta las bases para una segunda iteración donde se diseñen e implementen tanto los componentes que quedaron fuera del alcance de esta etapa, así como también los que estaban contemplados pero no pudieron implementarse.

Los principales aportes de este trabajos son:

\begin{itemize}
	\item Adopción de una metodología de desarrollo basada en control de versiones sistemática y robusta.
	\item Desarrollo de una arquitectura modular multicore que permite el intercambio de funcionalidades entre procesadores de forma flexible.
	\item Implementación de mecanismos de comunicación y sincronización entre procesadores, basados en interrupciones cruzadas y colas de mensajes.
	\item Desarrollo de 4 módulos con funcionalidad de adquisición de datos, almacenamientos de datos, HMI y control del sistema, respectivamente. 
	\item Implementación de mecanismos de control y autochequeo para aumentar la seguridad de operación del sistema.
	\item Documentación completa de ingeniería de detalle de los módulos desarrollados.
	\item Documentación completa de testing.
	\item Documentación completa de trazabilidad entre requerimientos, ingeniería de detalle y testing.
\end{itemize} 

\subsection{Técnicas útiles}

Resultaron particularmente útiles las técnicas de gestión de proyecto empleadas en la planificación de este trabajo.


\subsection{Metas alcanzadas}
\label{subsec:metasalcanzadas}

A los fines prácticos de terminar el trabajo en la fecha pactada, no fue posible implementar el control del anemómetro.  Esto no responde a una dificultad técnica sino a hechos imprevistos en la planificación que sobrecargaron la dedicación horaria a actividades ajenas al proyecto e imposibilitaron el cumplimiento del requisito asociado a este sensor.

%----------------------------------------------------------------------------------------
%	SECTION 2
%----------------------------------------------------------------------------------------
\section{Próximos pasos}

Acá se indica cómo se podría continuar el trabajo más adelante.
