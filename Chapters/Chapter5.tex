% Chapter Template

\chapter{Conclusiones} % Main chapter title

\label{Chapter5} % Change X to a consecutive number; for referencing this chapter elsewhere, use \ref{ChapterX}

En este capítulo se destacan los principales aportes del trabajo realizado y se listan los logros obtenidos.  Asimismo, se documentan las técnicas que resultaron  útiles para la ejecución del proyecto.  Por otra parte, se deja constancia de las metas que no pudieron ser alcanzadas junto con las respectivas causas y se identifican las líneas de acción a futuro.
%----------------------------------------------------------------------------------------

%----------------------------------------------------------------------------------------
%	SECTION 1
%----------------------------------------------------------------------------------------

\section{Conclusiones generales }

%La idea de esta sección es resaltar cuáles son los principales aportes del trabajo realizado y cómo se podría continuar. Debe ser especialmente breve y concisa. Es buena idea usar un listado para enumerar los logros obtenidos.

En este trabajo se logró completar una primera iteración en el ciclo de diseño e implementación de un sistema embebido para el control de una estación de monitoreo de ruido ambiente submarino.  

Se pudo desarrollar un \textit{firmware} multicore modular que sienta las bases para una segunda iteración donde se diseñen e implementen tanto los componentes que quedaron fuera del alcance de esta etapa, así como también los que estaban contemplados pero no pudieron implementarse.

Los principales aportes de este trabajos son:

\begin{itemize}
	\item Adopción de una metodología de desarrollo basada en control de versiones sistemática y robusta.
	\item Desarrollo de una arquitectura modular multicore que permite el intercambio de funcionalidades entre procesadores de forma flexible.
	\item Implementación de mecanismos de comunicación y sincronización entre procesadores, basados en interrupciones cruzadas y colas de mensajes.
	\item Desarrollo de cuatro módulos con funcionalidad de adquisición de datos, almacenamientos de datos, HMI y control del sistema, respectivamente. 
	\item Implementación de mecanismos de control y autochequeo para aumentar la seguridad de operación del sistema.
	\item Documentación completa de ingeniería de detalle de los módulos desarrollados.
	\item Documentación completa de testing.
	\item Documentación completa de trazabilidad entre requerimientos, ingeniería de detalle y testing.
\end{itemize} 

\subsection{Técnicas útiles}
\label{subsec:tecnicas_utiles}

Resultaron particularmente útiles las técnicas de gestión de proyecto empleadas en la planificación de este trabajo.  Mediante el desglose del proyecto en tareas y la articulación de éstas en diagramas de \textit{Activity on Node} y \textit{Gantt}, se puedo definir una metodología de seguimiento y control que permitió detectar retrasos en el avance del proyecto. Se aplicaron medidas de mitigación \textit{ad-hoc} que consistieron en:

\begin{itemize}
	\item el incremento de la dedicación horaria destinada al proyecto,
	\item el empleo de funciones \textit{mock} o maquetadas para algunas características secundarias que pensaban implementarse.
	\item y en última instancia, la priorización de los requerimientos de fecha de finalización por sobre la cantidad de sensores operativos.
\end{itemize}  

%Este plan de acción neutralizó parcialmente los efectos negativos sobre el resultado del proyecto al costo de no cumplir .

Dentro de la gestión de riesgos no se contempló explícitamente la posibilidad de imprevistos como los que efectivamente se manifestaron.  Una análisis posterior de los hechos muestra que los retrasos responden principalmente a dos fuentes:

\begin{enumerate}
	\item un evento fortuito, único e irrepetible, en el ámbito laboral, que sobrecargó la dedicación horaria a tareas ajenas al proyecto durante gran parte de tiempo planificado para su realización e incluyó día de embarque.
	\item la incorporación de nuevas responsabilidades en el campo de la docencia.
\end{enumerate}

La primera fuente constituye un hecho aislado y, por su carácter de imprevisible, resulta razonable que haya quedado fuera del alcance de la gestión de riesgos.  

La segunda fuente implica un error de cálculo en la relación entre horas disponibles y la carga horaria de las tareas de docencia asumidas.  Esto último será tenido en cuenta como aprendizaje para futuros proyecto.

Asimismo, dentro de la gestión de riesgos se observa que de los riesgos originalmente previstos:

\begin{enumerate}
	\item No contar a tiempo con el anemómetro.
	\item No contar con drivers para controlar el termómetro digital.
	\item No contar con el becario PIDDEF a tiempo para que pueda participar en el proyecto.
	\item Pérdida, robo o destrucción total o parcial de la placa CIAA-NXP y/o el anemómetro
	\item Cancelación del proyecto PIDDEF del Ministerio de Defensa asociado.
\end{enumerate}

\noindent sólo se manifestó literalmente el riesgo número 3. En ese sentido, resultó adecuada la estrategia de planificar las tareas para ser llevadas a cabo sin la colaboración de un becario.  Asimismo, fue correcta la valoración de severidad y tasa de ocurrencia para los riesgos identificados.

Respecto al riesgo número 5, si bien el proyecto no fue cancelado, el hecho de contar con un subsidio trianual como el otorgado por el programa PIDDEF \citep{PIDDEF}, que haya coincidido con el cambio de gestión en la administración pública ocasionó retrasos y sobrecostos significativos.

\subsection{Metas alcanzadas}
\label{subsec:metasalcanzadas}

A los fines prácticos de terminar el trabajo en la fecha pactada, no fue posible implementar el control del anemómetro.  Esto no responde a una dificultad técnica sino a hechos imprevistos en la planificación que sobrecargaron la dedicación horaria a actividades ajenas al proyecto e imposibilitaron el cumplimiento del requisito asociado a este sensor como fuera mencionado en la subsección \ref{subsec:tecnicas_utiles}.

\textbf{METER TABLA DE REQs ALCANZADOS Y NO ALCANZADOS}

%----------------------------------------------------------------------------------------
%	SECTION 2
%----------------------------------------------------------------------------------------
\section{Próximos pasos}

Acá se indica cómo se podría continuar el trabajo más adelante.
