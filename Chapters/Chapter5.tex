% Chapter Template

\chapter{Conclusiones} % Main chapter title

\label{Chapter5} % Change X to a consecutive number; for referencing this chapter elsewhere, use \ref{ChapterX}

En este capítulo se destacan los principales aportes del trabajo realizado y se listan los logros obtenidos.  Asimismo, se documentan las técnicas que resultaron  útiles para la ejecución del proyecto.  Por otra parte, se deja constancia de las metas que no pudieron ser alcanzadas junto con las respectivas causas, de modo que sirva como experiencia para futuros proyectos.  Finalmente, se identifican las líneas de acción a futuro y posibles mejoras para continuar el trabajo más adelante.
%----------------------------------------------------------------------------------------

%----------------------------------------------------------------------------------------
%	SECTION 1
%----------------------------------------------------------------------------------------

\section{Conclusiones generales }

La idea de esta sección es resaltar cuáles son los principales aportes del trabajo realizado y cómo se podría continuar. Debe ser especialmente breve y concisa. Es buena idea usar un listado para enumerar los logros obtenidos.

\subsection{Técnicas útiles}

\subsection{Metas no alcanzadas}
\label{subsec:metasnoalcanzadas}

A los fines prácticos de terminar el trabajo en la fecha pactada, no fue posible implementar el control del anemómetro.  Esto no responde a una dificultad técnica sino a hechos imprevistos en la planificación que sobrecargaron la dedicación horaria a actividades ajenas al proyecto e imposibilitaron el cumplimiento del requisito asociado a este sensor.

%----------------------------------------------------------------------------------------
%	SECTION 2
%----------------------------------------------------------------------------------------
\section{Próximos pasos}

Acá se indica cómo se podría continuar el trabajo más adelante.
