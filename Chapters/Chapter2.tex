\chapter{Introducción Específica} % Main chapter title

\label{Chapter2}

%----------------------------------------------------------------------------------------
%	SECTION 1
%----------------------------------------------------------------------------------------
\section{Requerimientos}
\label{sec:requerimientos}

A continuación se enumeran los requerimientos del proyecto:

\begin{enumerate}
  \item Requerimientos de documentación:
   \begin{enumerate}
     \item Se debe generar un Memoria Técnica con la documentación de ingeniería detallada.
	   \item Se debe generar un documento de casos de prueba.
	 \end{enumerate}
	\item Requerimientos funcionales del sistema:
	\begin{enumerate}
		\item El sistema debe adquirir datos de un array de sensores de temperatura a intervalos regulares con un período de adquisición seleccionable.
		\item El sistema debe adquirir datos de un anemómetro a intervalos regulares con un período de adquisición seleccionable.
		\item El sistema debe almacenar los datos de temperatura y velocidad de viento adquiridas junto con una marca de tiempo identificatoria en un medio físico no volátil.
		\item El sistema debe poder operar con dos perfiles de consumo de energía, uno maximizando el desempeño y otro minimizando el consumo de energía.
		\item El sistema debe contar con una interfaz serie tipo HMI cableada que permita interactuar y realizar operaciones de configuración y mantenimiento.
	\end{enumerate}
	\item Requerimientos de verificación:
	\begin{enumerate}
		\item Se debe generar una matriz de trazabilidad entre la Memoria Técnica y los requerimientos.
		\item Se debe generar una matriz de trazabilidad entre las pruebas de integración y los requerimientos.
	\end{enumerate}
	\item Requerimientos de validación:
	\begin{enumerate}
	  \item Se debe generar una matriz de trazabilidad entre el documento de casos de prueba y los requerimientos.
  \end{enumerate}
\end{enumerate}


\section{Planificación}
\label{sec:plan}

La planificación completa del proyecto puede encontrarse publicada en la web del Laboratorio de Sistemas Embebidos de FIUBA [\textbf{insertar referencia}].

A los fines de facilitar la comprensión del trabajo realizado, se detallan en la tabla \ref{tab:planificacion} las etapas del proyecto junto con la cantidad de horas destinadas y los hitos a alcanzar en cada una de ellas.  Puede observarse que el proyecto insume 600 horas de trabajo en total.

\begin{table}[h]
	\centering
	\includegraphics[width=\textwidth]{./Figures/planificacion.png}
	\caption[Diagrama \textit{Etapas del proyecto}.]{Etapas principales del proyecto con el detalle de las horas planificadas y los hitos a alcanzar en cada una de ellas.}
	\label{tab:planificacion}
\end{table}

Para las etapas planificadas en la tabla \ref{tab:planificacion}, se realizó un desglose de tareas que puede verse esquemáticamente en el diagrama de \textit{Activity on Node} que se ilustra en la figura \ref{fig:AoN}.  Se utiliza el mismo código de colores para identificar las diferentes etapas del proyecto y las tareas planificadas que las componen. En el diagrama, los tiempos de duración de las tareas está expresado en horas. Asimismo, las tareas poseen un código único que será usado para realizar la trazabilidad de los requerimientos.

\begin{figure}[h]
	\centering
	\includegraphics[width=\textwidth]{./Figures/AoN.pdf}
	\caption[Diagrama \textit{Activity on Node}.]{Diagrama \textit{Activity on Node}.  En colores pueden distinguirse las diferentes etapas del proyecto: en amarillo la etapa de documentación y análisis preliminar; en azul la etapa de diseño e implementación; en verde la etapa de verificación y validación y en rojo el proceso de cierre. El tiempo t está expresado en horas.}
	\label{fig:AoN}
\end{figure}

\clearpage
\section{Metodologías}

En esta sección se describen los aspectos metodológicos relevantes que se aplicaron durante el desarrollo del trabajo.  

\subsection{Control de versiones}
\label{subsec:branching}

Se adoptó un modelo de desarrollo creado por Vincent Driessen llamado ``\textit{A successful Git branching model''}\footnote{\url{https://nvie.com/posts/a-successful-git-branching-model/}}.  El modelo está basado en la herramienta de control de versiones \textit{git} y consiste en un conjunto de procedimientos para ordenar y sistematizar el flujo de trabajo. Este modelo propone utilizar un repositorio considerado a los fines prácticos ``central'' (en git todos los repositorios son idénticos) llamado \textit{origin}.  Todos los desarrolladores trabajan contra este repositorio central con las operaciones típicas de \textit{push} y \textit{pop}.  

Adicionalmente, puede haber intercambios entre los repositorios de los distintos desarrolladores que formen un mismo equipo de trabajo. Estos intercambios pueden visualizarse en la figura \ref{fig:esquema-repos}, donde se esquematizan por un lado los posibles flujos de trabajo entre el repositorio \textit{origin} y los distintos desarrolladores y por el otro, entre los repositorios propios de cada desarrollador. 

\begin{figure}[h]
	\centering
	\includegraphics[width=.6\textwidth]{./Figures/centr-decentr@2x.png}
	\caption[Esquema del flujo de trabajo entre repositorios]{Esquema del flujo de trabajo entre repositorios\protect\footnotemark.}
	\label{fig:esquema-repos}
\end{figure}

\footnotetext{Imagen tomada de \url{https://nvie.com/img/centr-decentr@2x.png}.}

Para la elaboración de este trabajo, donde la codificación recayó principalmente sobre una sóla persona, no fueron habituales las operaciones contra un repositorio distinto de \textit{origin} implementado en \textit{github}. Sin embargo, se considera la experiencia de apropiación de la metodología de trabajo muy valiosa para la formación profesional ya que el autor de este trabajo no había tenido oportunidad de trabajar tan extensa y sistemáticamente con control de versiones previamente.

En cuanto a la estrategia de uso de ramas, siguiendo el modelo adoptado, se dispuso de dos ramas principales llamas \textit{master} y \textit{develop}.  En \textit{origin/master} sólo se incluyen \textit{commits} con versiones estables con capacidad de ser puestas en producción, es decir sobre el prototipo de manera que éste pueda operar satisfactoriamente.  En \textit{origin/develop} contiene los últimos cambios que integran las diferentes características ya logradas del código.

\subsection{Programación concurrente con Protothreads} 
\label{subsec:protothreads}

Los Protothreads son una abstracción de programación creada por Adam Dunkel\footnote{\url{http://dunkels.com/adam/pt/}} para implementar mecanismos de programación concurrente conocidos como multi-tarea cooperativa en sistemas embebidos con recursos limitados. Funcionan como hilos de ejecución sin \textit{stack} o co-rutinas y proveen mecanismos para bloquear la ejecución de una tarea sin que se produzca un cambio de contexto.  Esto permite un control de flujo secuencial sin máquinas de estado complejas o soporte multi-hilo completo en arquitecturas basadas en eventos \citep{dunkels06protothreads} \citep{dunkels05using}. 

En el presente trabajo, se hace uso de protothreads en la codificación del protocolo de comunicación 1-wire que se describe en la subsección \ref{subsec:1-wire}.