\chapter{Introducción Específica} % Main chapter title

\label{Chapter2}

%----------------------------------------------------------------------------------------
%	SECTION 1
%----------------------------------------------------------------------------------------
\section{Requerimientos}
\label{sec:requerimientos}

\begin{enumerate}
  \item Requerimientos de documentación:
   \begin{enumerate}
     \item Se debe generar un Memoria Técnica con la documentación de ingeniería detallada.
	   \item Se debe generar un documento de casos de prueba.
	 \end{enumerate}
	\item Requerimientos funcionales del sistema:
	\begin{enumerate}
		\item El sistema debe adquirir datos de un array de sensores de temperatura a intervalos regulares con un período de adquisición seleccionable.
		\item El sistema debe adquirir datos de un anemómetro a intervalos regulares con un período de adquisición seleccionable.
		\item El sistema debe almacenar los datos de temperatura y velocidad de viento adquiridas junto con una marca de tiempo identificatoria en un medio físico no volátil.
		\item El sistema debe poder operar con dos perfiles de consumo de energía, uno maximizando el desempeño y otro minimizando el consumo de energía.
		\item El sistema debe contar con una interfaz serie tipo HMI cableada que permita interactuar y realizar operaciones de configuración y mantenimiento.
	\end{enumerate}
	\item Requerimientos de verificación:
	\begin{enumerate}
		\item Se debe generar una matriz de trazabilidad entre la Memoria Técnica y los requerimientos.
		\item Se debe generar una matriz de trazabilidad entre las pruebas de integración y los requerimientos.
	\end{enumerate}
	\item Requerimientos de validación:
	\begin{enumerate}
	  \item Se debe generar una matriz de trazabilidad entre el documento de casos de prueba y los requerimientos.
  \end{enumerate}
\end{enumerate}


\subsection{Este es el título de una subsección}
\label{subsec:ejemplo}

Se recomienda no utilizar \textbf{texto en negritas} en ningún párrafo, ni tampoco texto \underline{subrayado}. En cambio sí se sugiere utilizar \textit{texto en cursiva} donde se considere apropiado.

Se sugiere que la escritura sea impersonal. Por ejemplo, no utilizar ``el diseño del firmware lo hice de acuerdo con tal principio'', sino ``el firmware fue diseñado utilizando tal principio''. En lo posible hablar en tiempo pasado, ya que la memoria describe un trabajo que ya fue realizado.

Se recomienda no utilizar una sección de glosario sino colocar la descripción de las abreviaturas como parte del mismo cuerpo del texto. Por ejemplo, RTOS (\textit{Real Time Operating System}, Sistema Operativo de Tiempo Real) o en caso de considerarlo apropiado mediante notas a pie de página.

Si se desea indicar alguna página web utilizar el siguiente formato de referencias bibliográficas, dónde las referencias se detallan en la sección de bibliografía de la memoria,utilizado el formato establecido por IEEE en \citep{IEEE:citation}. Por ejemplo, ``el presente trabajo se basa en la plataforma EDU-CIAA-NXP, la cual se describe en detalle en \citep{CIAA}''.

\subsection{Figuras} 

Al insertar figuras en la memoria se deben considerar determinadas pautas. Para empezar, usar siempre tipografía claramente legible. Luego, tener claro que es incorrecto escribir por ejemplo esto: ``El diseño elegido es un cuadrado, como se ve en la siguiente figura:''

\begin{figure}[h]
\centering
\includegraphics[scale=.35]{./Figures/cuadradoAzul.png}
\end{figure}

La forma correcta de utilizar una figura es la siguiente: ``Se eligió utilizar un cuadrado azul para el logo, el cual se ilustra en la figura \ref{fig:cuadradoAzul}''.

\begin{figure}[h]
	\centering
	\includegraphics[scale=.35]{./Figures/cuadradoAzul.png}
	\caption{Ilustración del cuadrado azul que se eligió para el diseño del logo.}
	\label{fig:cuadradoAzul}
\end{figure}

El texto de las figuras debe estar siempre en español, excepto que se decida reproducir una figura original tomada de alguna referencia. En ese caso la referencia de la cual se tomó la figura debe ser indicada en el epígrafe de la figura e incluida como una nota al pie, como se ilustra en la figura \ref{fig:palabraIngles}.

\begin{figure}[h!]
	\centering
	\includegraphics[scale=.25]{./Figures/word.jpeg}
	\caption{Imagen tomada de la página oficial del procesador\protect\footnotemark.}
	\label{fig:palabraIngles}
\end{figure}

\footnotetext{\url{https://goo.gl/images/i7C70w}}


La figura y el epígrafe deben conformar una unidad cuyo significado principal pueda ser comprendido por el lector sin necesidad de leer el cuerpo central de la memoria. Para eso es necesario que el epígrafe sea todo lo detallado que corresponda y si en la figura se utilizan abreviaturas entonces aclarar su significado en el epígrafe o en la misma figura.

\begin{figure}[h]
	\centering
	\includegraphics[scale=.4]{./Figures/questionMark.png}
	\caption{El lector no sabe por qué de pronto aparece esta figura.}
	\label{fig:questionMark}
\end{figure}

Nunca colocar una figura en el documento antes de hacer la primera referencia a ella, como se ilustra con la figura \ref{fig:questionMark}, porque sino el lector no comprenderá por qué de pronto aparece la figura en el documento, lo que distraerá su atención.

\subsection{Tablas}

Para las tablas utilizar el mismo formato que para las figuras, sólo que el epígrafe se debe colocar arriba de la tabla, como se ilustra en la tabla \ref{tab:peces}. Observar que sólo algunas filas van con líneas visibles y notar el uso de las negritas para los encabezados.  La referencia se logra utilizando el comando \verb|\ref{<label>}| donde label debe estar definida dentro del entorno de la tabla.

\begin{verbatim}
\begin{table}[h]
	\centering
	\caption[caption corto]{caption largo más descriptivo}
	\begin{tabular}{l c c}    
		\toprule
		\textbf{Especie}       & \textbf{Tamaño}  & \textbf{Valor aprox.}\\
		\midrule
		Amphiprion Ocellaris	  & 10 cm 			& \$ 6.000 \\		
		Hepatus Blue Tang      & 15 cm			 & \$ 7.000 \\
		Zebrasoma Xanthurus    & 12 cm			 & \$ 6.800 \\
		\bottomrule
		\hline
	\end{tabular}
	\label{tab:peces}
\end{table}
\end{verbatim}

\begin{table}[h]
	\centering
	\caption[caption corto]{caption largo más descriptivo}
	\begin{tabular}{l c c}    
		\toprule
		\textbf{Especie} 	 & \textbf{Tamaño}  & \textbf{Valor aprox.}  \\
		\midrule
		Amphiprion Ocellaris	 & 10 cm 			& \$ 6.000 \\		
		Hepatus Blue Tang	 & 15 cm				& \$ 7.000 \\
		Zebrasoma Xanthurus	 & 12 cm				& \$ 6.800 \\
		\bottomrule
		\hline
	\end{tabular}
	\label{tab:peces}
\end{table}

En cada capítulo se debe reiniciar el número de conteo de las figuras y las tablas, por ejemplo, Fig. 2.1 o Tabla 2.1, pero no se debe reiniciar el conteo en cada sección. Por suerte la plantilla se encarga de esto por nosotros.

\subsection{Ecuaciones}
\label{sec:Ecuaciones}

Al insertar ecuaciones en la memoria estas se deben numerar de la siguiente forma:

\begin{equation}
	\label{eq:metric}
	ds^2 = c^2 dt^2 \left( \frac{d\sigma^2}{1-k\sigma^2} + \sigma^2\left[ d\theta^2 + \sin^2\theta d\phi^2 \right] \right)
\end{equation}
                                                        
Es importante tener presente que en el caso de las ecuaciones estas pueden ser referidas por su número, como por ejemplo ``tal como describe la ecuación \ref{eq:metric}'', pero también es correcto utilizar los dos puntos, como por ejemplo ``la expresión matemática que describe este comportamiento es la siguiente:''

\begin{equation}
	\label{eq:schrodinger}
	\frac{\hbar^2}{2m}\nabla^2\Psi + V(\mathbf{r})\Psi = -i\hbar \frac{\partial\Psi}{\partial t}
\end{equation}

Para las ecuaciones se debe utilizar un tamaño de letra equivalente al utilizado para el texto del trabajo, en tipografía cursiva y preferentemente del tipo Times New Roman o similar. El espaciado antes y después de cada ecuación es de aproximadamente el doble que entre párrafos consecutivos del cuerpo principal del texto. Por suerte la plantilla se encarga de esto por nosotros.

Para generar la ecuación \ref{eq:metric} se utilizó el siguiente código:

\begin{verbatim}
\begin{equation}
	\label{eq:metric}
	ds^2 = c^2 dt^2 \left( \frac{d\sigma^2}{1-k\sigma^2} + 
	\sigma^2\left[ d\theta^2 + 
	\sin^2\theta d\phi^2 \right] \right)
\end{equation}
\end{verbatim}

Y para la ecuación \ref{eq:schrodinger}:

\begin{verbatim}
\begin{equation}
	\label{eq:schrodinger}
	\frac{\hbar^2}{2m}\nabla^2\Psi + V(\mathbf{r})\Psi = 
	-i\hbar \frac{\partial\Psi}{\partial t}
\end{equation}

\end{verbatim}